\begin{abstract}
This document describes version 2.0 of \OM: a standard for the representation and
communication of mathematical objects.

This version clarifies and extends \OM 1.1 \cite{OM_1.1}.  \OM allows the \emph{meaning}
 of an object to be encoded rather than just a visual representation.  It is designed to
 allow the free exchange of mathematical objects between software systems and human
 beings.  On the worldwide web it is designed to allow mathematical expressions embedded
 in web pages to be manipulated and computed with in a meaningful and correct way.  It is
 designed to be machine-generatable and machine-readable, rather than written by hand.

The \OM Standard is the official reference for the \OM language and has been approved by
the \OM Society.  It is not intended as an introductory document or a user's guide, for
the latest available material of this nature please consult the \OM web-site at
\url{http://www.openmath.org}.


This document includes an overview of the \OM architecture, an abstract description of \OM
objects and two mechanisms for producing concrete encodings of such objects.  The first,
in \XML, is designed primarily for use on the web, in documents, and for applications
which want to mix \OM as a content representation with MathML as a presentation format.
The second, a binary format, is designed for applications which wish to exchange very
large objects, or a lot of data as efficiently as possible.  This document also includes a
description of Content Dictionaries - the mechanism by which the meaning of a symbol in
the \OM language is encoded, as well as an XML encoding for them.  Finally it includes
guidelines for the development of \OM-compliant applications. Further background on \OM
and guidelines for its use in applications may be found in the accompanying Primer
\cite{OM_primer}.
\end{abstract}

%%% Local Variables:
%%% mode: latex
%%% TeX-master: "omstd20"
%%% End:
