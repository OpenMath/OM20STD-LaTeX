\chapter{Content Dictionaries}\label{cha_cd}
  
In this chapter we give a brief overview of Content Dictionaries before explicitly stating
their functionality and encoding.

\section{Introduction}\label{sec_cd_summary}
    
Content Dictionaries (CDs) are central to the \OM philosophy of transmitting mathematical
information. It is the \OM Content Dictionaries which actually hold the meanings of the
objects being transmitted.
    
For example if application $A$ is talking to application $B$, and sends, say, an equation
involving multiplication of matrices, then $A$ and $B$ must agree on what a matrix is, and
on what matrix multiplication is, and even on what constitutes an equation. All this
information is held within some Content Dictionaries which both applications agree upon.
    
A \emph{ Content Dictionary} holds the meanings of (various) mathematical
\textquote{words}. These words are \OM basic objects referred to as \emph{symbols} in
\ref{sec_omabs}.
    
With a set of symbol definitions (perhaps from several Content Dictionaries), $A$ and $B$
can now talk in a common \textquote{language}.
    
It is important to stress that it is not Content Dictionaries themselves which are being
transmitted, but some \textquote{mathematics} whose definitions are held within the
Content Dictionaries. This means that the applications must have already agreed on a set
of Content Dictionaries which they \textquote{understand} (i.e., can cope with to some
degree).
    
In many cases, the Content Dictionaries that an application understands will be constant,
and be intrinsic to the application's mathematical use. However the above approach can
also be used for applications which can handle every Content Dictionary (such as an \OM
parser, or perhaps a typesetting system), or alternatively for applications which
understand a changeable number of Content Dictionaries (perhaps after being sent Content
Dictionaries in some way).
    
The primary use of Content Dictionaries is thought to be for designers of Phrasebooks, the
programs which translate between the \OM mathematical object and the corresponding (often
internal) structure of the particular application in question. For such a use the Content
Dictionaries have themselves been designed to be as readable and precise as possible.
    
Another possible use for \OM Content Dictionaries could rely on their automatic
comprehension by a machine (e.g., when given definitions of objects defined in terms of
previously understood ones), in which case Content Dictionaries may have to be passed as
data. Towards this end, a Content Dictionary has been written which contains a set of
symbols sufficient to represent any other Content Dictionary. This means that Content
Dictionaries may be passed in the same way as other (\OM) mathematical data.
    
Finally, the syntax of the reference encoding for Content Dictionaries has been designed
to be relatively easy to learn and to write, and also free from the need for any
specialist software. This is because it is acknowledged that there is an enormous amount
of mathematical information to represent, and so most Content Dictionaries are written by
\textquote{ordinary} mathematicians, encoding their particular fields of expertise.  A
further reason is that the mathematics conveyed by a specific Content Dictionary should be
understandable independently of any application.
    
The key points from this section are:
\begin{itemize}
\item Content Dictionaries should be readable and precise to help Phrasebook designers,
\item Content Dictionaries should be readily write-able to encourage widespread use,
\item It ought to be possible for a machine to understand a Content Dictionary to some
  degree.
\end{itemize}
  
\section{Abstract Content Dictionaries}\label{sect_func}
    
In this section we define the abstract structure of Content Dictionaries.
    
A Content Dictionary consists of the following mandatory pieces of information:
\begin{enumerate}
\item A \emph{name} corresponding to the rules described in \ref{sec_names}.
\item A \emph{description} of the Content Dictionary.
\item A \emph{revision date}, the date of the last change to the Content Dictionary.
  Dates should be stored in the ISO-compliant format YYYY-MM-DD, e.g. 1966-02-03.
\item A \emph{review date}, a date until which the content dictionary is guaranteed to
  remain unchanged.
\item A \emph{version number} which consists of a major and minor part (see
  \ref{sec_version}).
\item A \emph{status}, as described in \ref{sec_status}.
\item A \emph{CD base} which, when combined with the CD name, forms a unique identifier
  for the Content Dictionary. It may or may not refer to an actual location from which it
  can be retrieved.
\item A series of one or more \emph{symbol definitions} as described below.
\end{enumerate}
A symbol definition consists of the following pieces of information:
\begin{enumerate}
\item A mandatory \emph{name} corresponding to the rules described in \ref{sec_names}.
\item A mandatory \emph{description} of the symbol, which can be as formal or informal as
  the author likes.
\item An optional \emph{role} as described in \ref{sec_roles}.
\item Zero or more \emph{commented mathematical properties} which are
  mathematical properties of the symbol expressed in a mechanism other than \OM.
\item Zero or more \emph{formal mathematical properties} which are
  mathematical properties of the symbol expressed in \OM.  Note that it is common for
  commented and formal mathematical properties to be introduced in pairs, with the former
  describing the latter.
	  
  A Formal Mathematical Property may be given an optional \emph{kind} attribute.  An
  author of a Content Dictionary may use this to indicate whether, for example, the
  property provides an algorithm for evaluation of the concept it is associated with.  At
  present no fixed scheme is mandated for how this information should be encoded or used
  by an application.
\item Zero or more \emph{examples} which are intended to demonstrate the use of the symbol
  within an \OM object.
\end{enumerate}
Some pieces of information which might logically be thought to be part of a Content
Dictionary, such as the types or signatures of symbols, are better represented externally.
This allows for new variants to be associated with Content Dictionaries without the
Dictionaries themselves being changed.  A model for signatures is given in \ref{sigfiles}.
    
Content Dictionaries may be grouped into \emph{CD Groups}. These groups
allow applications to easily refer to collections of Content Dictionaries. One particular
CDGroup of interest is the \textquote{MathML CDGroup}. This group consists of the
collection of core Content Dictionaries that is designed to have the same semantic scope
as the content elements of MathML~2~\cite{MathML_2003}.  \OM objects built from symbols
that come from Content Dictionaries in this CDGroup may be expected to be easily
transformed between \OM and MathML encodings.  The detailed structure of a CDGroup is
described in \ref{ssec_cdgroups} below.
    
\subsubsection{Content Dictionary Status}\label{sec_status}

The status of a Content Dictionary can be either
\begin{itemize}
\item \lstinline|official|: approved by the \OM Society according to the procedure
  outlined in \ref{cdapprove};
\item \lstinline|experimental|: under development and thus liable to change;
\item \lstinline|private|: used by a private group of \OM users;
\item \lstinline|obsolete|: an obsolete Content Dictionary kept only for archival
  purposes.
\end{itemize}
 
\subsubsection{Content Dictionary Version Numbers}\label{sec_version}
      
A version number must consist of two parts, a major version and a revision, both of which
should be non-negative integers.  In CDs that do not have status \emph{experimental}, the
version number should be a positive integer.
      
Unless a CD has status \emph{experimental}, no changes should ever be introduced that
invalidate objects built with previous versions.  Any change that influences phrasebook
compliance, like adding a new symbol to a Content Dictionary, is considered a major change
and should be reflected by an increase in the major version number. Other changes, like
adding an example or correcting a description, are considered minor changes. For minor
changes the version number is not changed, but an increase should be made to the revision
number.  Note that a change such as removing a symbol should not be made unless the CD has
status \emph{experimental}.  Should this be required then a new CD with a different name
should be produced so as not to invalidate existing objects.
      
When the major version number is increased, the revision number is normally reset to zero.
      
As detailed in \ref{cha_comp}, \OM compliant applications state which versions of which
CDs they support.
  
\section{The Reference Encoding for Content Dictionaries}\label{sec_xml_cd}
    
The reference encoding of Content Dictionaries are as \XML documents.  A valid Content
Dictionary document should conform to the Relax NG Schema for Content Dictionaries given
in \ref{sec_cd_schema}.
    
An example of a complete Content Dictionary is given in Appendix~\ref{app_cdcd}, which is
the \lstinline|Meta| Content Dictionary for describing Content Dictionaries themselves. A
more typical Content Dictionary is given in \ref{arith1.ocd}, the \lstinline|arith1|
Content Dictionary for basic arithmetic functions.

    
\subsubsection{The Relax NG Schema for Content Dictionaries}\label{sec_cd_schema}

\lstinputlisting{omcd2.rnc}

\subsubsection{Further Description of the CD Schema}\label{sect_pcdata}

We now describe the elements used in the above schema in terms of the abstract description
of CDs in \ref{sect_func}.  Unless stated otherwise information is encoded as the content
of the relevant element.

\begin{description}
\item[\lstinline|CDName|] The name of the Content Dictionary.
\item[\lstinline|Description|] The text occurring in the \lstinline|Description| element
  is used to give a description of the enclosing element, which could be a symbol or the
  entire Content Dictionary. The content of this element can be any \XML text.
\item[\lstinline|CDReviewDate|] The review date of the Content Dictionary.
\item[\lstinline|CDDate|] The revision date of this version of the Content Dictionary.
\item[\lstinline|CDVersion|] The major version number of the CD.
\item[\lstinline|CDRevision|] The minor version number of the CD.
\item[\lstinline|CDStatus|] The status of the Content Dictionary.
\item[\lstinline|CDBase|] The CD base of the CD.
\item[\lstinline|CDURL|] The text occurring in the \lstinline|CDURL| element should be a
  valid URL where the source file for the Content Dictionary encoding can be found (if it
  exists). The filename should conform to ISO 9660~\cite{iso9660}.
\item[\lstinline|CDUses|] The content of this element should be a series of
  \lstinline|CDName| elements, each naming a Content Dictionary used in the
  \lstinline|Example| and \lstinline|FMP|s of the current Content Dictionary. This
  element is optional and deprecated since the information can easily be extracted
  automatically.
\item[\lstinline|CDComment|] The content of this element should be text that does not
  convey any crucial information concerning the current Content Dictionary. It can be used
  in the Content Dictionary header to report the author of the Content Dictionary and to
  log change information. In the body of the Content Dictionary, it can be used to attach
  extra remarks to certain symbols.
\item[\lstinline|CDDefinition|] The element which contains the definition of an
  individual symbol.
\item[\lstinline|Name|]The name of a symbol.
\item[\lstinline|Role|] The role of a symbol: it must be one of \emph{binder},
  \emph{attribution}, \emph{semantic-attribution}, \emph{error}, \emph{application}, or
  \emph{constant}.
\item[\lstinline|Example|] The text occurring in the \lstinline|Example| element is used
  to give examples of the enclosing symbol, and can be any \XML text. In addition to text
  the element may contain examples as \XML encoded \OM, inside \lstinline|OMOBJ|
  elements.  Note that \lstinline|Examples| must be with respect to some symbol and
  cannot be \textquote{loose} in the Content Dictionary.
\item[\lstinline|CMP|] A Commented Mathematical Property.
\item[\lstinline|FMP|] A Formal Mathematical Property.  It may take an optional
  \lstinline|kind| attribute.
\end{description}

\section{Additional Information}\label{addfiles}

Content Dictionaries contain just one part of the information that can be associated to a
symbol in order to define its meaning and its functionality. \OM Signature dictionaries,
CDGroups, and possibly collections of extra mathematical properties, are used to convey
the different aspects that as a whole make up a mathematical definition.

\subsection{Signature Dictionaries}\label{sigfiles}

\OM may be used with any type system. One just needs to produce a Content Dictionary which
gives the constructors of the type system, and then one may build \OM objects representing
types in the given type system. These are typically associated with \OM objects via the
\OM \lstinline|attribution| constructor.

A Small Type System, called STS, has been designed to give semi-formal signatures to \OM
symbols and is documented in~\cite{OM_D132c}.  The signature file given in
\ref{arith1.sts} is based on this formalism. Using the same mechanism, \cite{OMD132b}
shows how pure type systems can also be employed to assign types to \OM symbols.

\subsubsection{Abstract Specification  of a Signature Dictionary}\label{sect_sigpcdata}

Signature dictionaries have a header which specifies the type system being used and the
Content Dictionary containing the symbols for which signatures are being given. Each
signature takes the form of an \OM object in an appropriate encoding.

\begin{enumerate}
\item A \emph{type definition}: the name of the Content Dictionary or of the CDGroup
  (cfg. \ref{ssec_cdgroups}) that represents the type system being used.
\item A \emph{CD name}: the name of the CD for which signatures are being defined.
\item A \emph{review date} as defined in \ref{sect_func}.
\item A \emph{status}: as defined in \ref{sect_func}.
\item A series of \emph{signatures} which are \OM objects in some encoding.  The objects
  must represent types as defined by the type definition.
\end{enumerate}

\subsubsection{A Relax NG Schema for a Signature Dictionary}\label{sect_sigschema}

The following is a reference encoding of a signature dictionary,
designed to be used with Content Dictionaries in the \XML encoding.

\lstinputlisting{omcdsig2.rnc}

\subsubsection{Examples}\label{sect_sigex}

An example of a signature dictionary for the type system STS and the \lstinline|arith1|
Content Dictionary is given in \ref{arith1.sts}. Each signature entry is similar to the
following one for the \OM symbol \lstinline|<OMS cd="arith1" name="plus"/>|.

\begin{lstlisting}
<Signature name="plus">
<OMOBJ version="2.0">
 <OMA>
  <OMS name="mapsto" cd="sts"/>
  <OMA>
   <OMS name="nassoc" cd="sts"/> 
   <OMV name="AbelianSemiGroup"/>
  </OMA>
  <OMV name="AbelianSemiGroup"/>
 </OMA>
</OMOBJ>
</Signature>
\end{lstlisting}

\subsection{CDGroups}\label{ssec_cdgroups}

The CD Group mechanism is a convenience mechanism for identifying collections of CDs.  A
CD Group file is an \XML document used in the (static or dynamic) negotiation phase where
communicating applications declare and agree on the Content Dictionaries which they
process.  It is a complement, or an alternative, to the individual declaration of Content
Dictionaries understood by an application.  Note that CD Groups do \emph{not} affect the
\OM objects themselves.  Symbols in an object always refer to content dictionaries, not
groups.

For an application to declare that it \textquote{understands CDGroup G} is exactly
equivalent to, and interchangeable with, the declaration that it \textquote{understands}
Content Dictionaries $x_1$, $x_2$, \ldots $x_n$, where $x_1$, \ldots, $x_n$ are the
members of CDGroup G.

\subsubsection{The Specification of CDGroups}\label{sec_dtd_cdg}

CDGroups are \XML documents, hence  a valid  CDGroup
 should 
\begin{itemize}
\item be valid according to the schema given in \ref{fig_cdgroup.dtd},
\item adhere to the extra conditions on the content of the elements
  given in \ref{sect_cdgpcdata}.
\end{itemize}

Apart from some header information such as \lstinline|CDGroupName| and
\lstinline|CDGroup| version, a CDGroup is simply an unordered list of
CDs, identified by name and optionally version number and URL.

\begin{figure}\centering
  \lstinputlisting{omcdgroup2.rnc}
  \caption{Relax NG Specification of CDGroups}\label{fig_cdgroup.dtd}
\end{figure}


\subsubsection{Further Requirements of a CDGroup}\label{sect_cdgpcdata}

The notion of being a valid CDGroup implies that the following requirements on the content
of the elements described by the schema given in \ref{sect_sigschema} are also met.

\begin{description}
\item[\lstinline|CDGroup|] The \XML element \lstinline|CDGroup| is the outermost element
  in a CDGroup document.
\item[\lstinline|CDGroupName|] The text occurring in the \lstinline|CDGroupName| element
  corresponds to the name of the CDGroup. For the syntactical requirements, see
  \lstinline|CDName| in \ref{sect_pcdata}.
\item[\lstinline|CDGroupVersion|]
\item[\lstinline|CDGroupRevision|] The text occurring in these elements contains the
  major and minor version numbers of the CDGroup.
\item[\lstinline|CDGroupURL|] The text occurring in the \lstinline|CDGroupURL| element
  identifies the location of the CDGroup file, not necessarily of the member Content
  Dictionaries. For the syntactical requirements, see \lstinline|CDURL| in
  \ref{sect_pcdata}.
\item[\lstinline|CDGroupDescription|] The text occurring in the
  \lstinline|CDGroupDescription| element describes the mathematical area of the CDGroup.
\item[\lstinline|CDGroupMember|] The \XML element \lstinline|CDGroupMember| encloses the
  data identifying each member of the CDGroup.
\item[\lstinline|CDName|] The text occurring in the \lstinline|CDName| element
  corresponds to the name of a Content Dictionary in the CDGroup. For the syntactical
  requirements, see \lstinline|CDName| in \ref{sect_pcdata}.
\item[\lstinline|CDVersion|] The text occurring in the \lstinline|CDVersion| element
  identifies which version of the Content Dictionary is to be taken as member of the
  CDGroup. This element is optional. In case it is missing, the latest version is the one
  included in the CDGroup.  For the syntactical requirements, see \lstinline|CDVersion|
  in \ref{sect_pcdata}.
\item[\lstinline|CDURL|] The text occurring in the \lstinline|CDURL| element identifies
  the location of the Content Dictionary to be taken as member of the CDGroup. This
  element is optional. In case it is missing, the location of the CDGroup identified by
  the element \lstinline|CDGroupURL| is assumed.  For the syntactical requirements, see
  \lstinline|CDURL| in \ref{sect_pcdata}.
\item[\lstinline|CDComment|] See \lstinline|CDComment| in \ref{sect_pcdata}.
\end{description}

\section{Content Dictionaries Reviewing Process}\label{cdapprove}

The \OM Society is responsible for implementing a review and referee process to assess the
accuracy of the mathematical content of Content Dictionaries.  The status (see
\lstinline|CDStatus|) and/or the version number (see \lstinline|CDVersion| ) of a
Content Dictionary may change as a result of this review process.

%%% Local Variables:
%%% mode: latex
%%% TeX-master: "omstd20"
%%% End:
