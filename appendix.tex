\begin{appendix}
\chapter{CD Files}\label{app_cdfiles}

\section{The \texttt{meta} Content Dictionary}\label{app_cdcd}
\lstinputlisting{meta.ocd}

\section{The  \texttt{arith1} Content Dictionary File}\label{arith1.ocd}
\lstinputlisting{arith1.ocd}

\section{The  \texttt{arith1} STS Signature File}\label{arith1.sts}
\lstinputlisting{arith1.sts}

\section{The  \texttt{MathML} CDGroup}\label{mathml.cdg}
\lstinputlisting{mathml.cdg}

\section{The \texttt{error} Content Dictionary}\label{errorcd}
\lstinputlisting{error.ocd}

\chapter{\OM Schema in Relax NG XML Syntax (Normative)}\label{app_openmath.rng}

This is the Relax NG Schema described in \ref{sec_xml} expressed according to the Relax NG
XML Syntax.

\lstinputlisting{openmath2.rng}

\chapter{Restricting the \OM Schema (Non-Normative)}\label{app_relaxrestricted}
Relax NG allows one to state constraints such as \emph{if the cd attribute of OMS is
arith1 then the name attribute must be one of lcm, gcd, plus etc.} Thus it is
easy to use a stylesheet to generate for any given CD, a Relax NG schema that expresses
the constraint that an \lstinline|OMS| naming that CD must only use symbols defined in
the specified dictionary.  Similarly it is possible to use the \emph{role} information
contained in the CD to restrict which symbols can be the first child of an
\lstinline|OMBIND| or the odd-numbered children of an \lstinline|OMATP|.
  
The modularisation mechanisms of Relax NG then allow one to include these schema for all
the CDs that you want to allow and, for example, to replace the regexp-based validation of
the \lstinline|OMS| attributes by explicit lists of allowed CD names, and for each CD
Name, a list of allowed symbol names.
  
    For example, a CD-specific Relax NG Schema for the arith1 CD shown in
    \ref{arith1.ocd} would look like:
\begin{lstlisting}
<g?xml version="1.0" encoding="UTF-8"?>
<grammar xmlns="http://relaxng.org/ns/structure/1.0" datatypeLibrary="">
  <define name="cd.attlist.OMS" combine="choice">
    <attribute name="cd">
      <value type="string">arith1</value>
    </attribute>
    <attribute name="name">
      <choice>
        <value type="string">lcm</value>
        <value type="string">gcd</value>
        <value type="string">plus</value>
        <value type="string">unary_minus</value>
        <value type="string">minus</value>
        <value type="string">times</value>
        <value type="string">divide</value>
        <value type="string">power</value>
        <value type="string">abs</value>
        <value type="string">root</value>
        <value type="string">sum</value>
        <value type="string">product</value>
      </choice>
    </attribute>
  </define>
</grammar>
\end{lstlisting}
or, using the Relax NG compact syntax:
\begin{lstlisting}
  cd.attlist.OMS |= 
  attribute cd {string "arith1" },
  attribute name {
  string "lcm" |
  string "gcd" |
  string "plus" |
  string "unary_minus" |
  string "minus" |
  string "times" |
  string "divide" |
  string "power" |
  string "abs" |
  string "root" |
  string "sum" |
  string "product" }
\end{lstlisting}

 To build a schema that allows only symbols from arith1 we just
  need to include the \OM schema described in \ref{ssec_xml}, override the attribute declarations for
  OMS, and then include the schema for arith1.  For example:
\begin{lstlisting}
<?xml version="1.0" encoding="UTF-8"?>
    <grammar xmlns="http://relaxng.org/ns/structure/1.0">
      <include href="openmath.rng">
        <define name="attlist.OMS">
          <ref name="cd.attlist.OMS"/>
        </define>
      </include>
      <include href="arith1.rng"/>
    </grammar>
\end{lstlisting}
    or, in the compact syntax:
\begin{lstlisting}
  include "openmath.rnc" {
    attlist.OMS = cd.attlist.OMS}
  include "arith1.rnc"
\end{lstlisting}
Using this approach it is possible to include as many files as required.

\chapter{\OM Schema in XSD Syntax (Non-Normative)}\label{app_xsd}
This is an XSD Schema generated from the Relax NG Schema described in \ref{sec_xml}.
  
\lstinputlisting{openmath2.xsd}

\chapter{\OM DTD (Non-Normative)}\label{app_dtd}

This is a DTD generated from the Relax NG Schema described in \ref{sec_xml}.  Note that we
cannot express the fact that the \lstinline|OMFOREIGN| element can contain any
well-formed XML, so we have simply restricted it to contain any XML defined in the DTD.
  
\lstinputlisting{openmath2.dtd}

\chapter{Changes between \OM~1.1 and \OM~2 (Non-Normative)}\label{app_whats_new}
  
In this appendix we describe the major changes that occurred between version 1.1 and
version 2 of the \OM standard. All changes to the encodings and content dictionaries have
been designed to be backward compatible, in other words all existing \OM objects and
Content Dictionaries are still valid in \OM~2.  On the other hand an existing \OM~1.1
application may not be able to process \OM~2 objects.
  
\section{Changes to the Formal Definition of Objects}\label{chgformal}
    
Additional features of abstract objects have been introduced:

\begin{itemize}
\item \OM symbols have an optional role qualifier which restricts the place where they may
  occur within compound objects.  Although part of the abstract description of a symbol
  this information is intended to be stored in the CD.  In the \XML encoding it may be
  used to provide a more restricted schema leading to tighter validation.
\item In addition to their \emph{name} and \emph{cd} properties, symbols now have an
  optional \emph{cdbase} property.  This can be used to disambiguate between two CDs which
  are produced independently but have the same name, and can also be used to produce a
  canonical URI for any \OM symbol for use in frameworks such as RDFS or MathML which need
  one.
\item An \OM object may be attributed with a non-\OM object using the new \emph{foreign}
  constructor.  This allows an \XML-encoded \OM object to be attributed with appropriate
  Presentation MathML, for example, or a base-64 encoded MPEG file of its aural rendering.
\item In addition, an \OM error object may take as its arguments non-\OM objects wrapped
  in the new \emph{foreign} constructor.
\item The new role property can be used to indicate that a symbol is an
  \emph{attribution}, in which case an application may ignore or remove it, or a
  \emph{semantic attribution} in which case removing it is no longer
  guaranteed to produce an equivalent object.
\item Restrictions on the names of symbols, variables and content dictionaries have been
  relaxed to be compatible with XML and to be less Anglo-Saxon.
\end{itemize}
  
\section{Changes to the encodings}\label{chgenc}
    
The \OM version 2 standard still mandates two encodings: \XML and binary. The \XML
encoding in particular has been updated to reflect the latest development of \XML and is
now a full \XML application.  Version 2 encodings are backward compatible with version 1.1
encodings.
\begin{itemize}
\item Both encodings have been updated to support the changes to the model of abstract
  objects described above.
\item Encodings support internal and external sharing of objects
\item An optional attribute defining the version of the encoding can be specified for the
  encoded object
\item The \XML encoding in version 2 is defined by a Relax NG schema and the mandated
  character-based grammar of version 1 has been removed, while the DTD has been relegated
  to an Appendix.
\item The symbolic values \lstinline|INF|, \lstinline|-INF| and \lstinline|NaN| have
  been added to the decimal attribute of an \lstinline|OMF| in the \XML encoding, and
  guidelines on the interpretation of \lstinline|NaN|s added to the compliance section.
\item The Binary encoding has been extended to support the streaming of objects.
\end{itemize}
    
\section{Changes to Content Dictionaries}\label{chgcd}
    
\begin{itemize}
\item In \OM version 2 Content Dictionaries are defined in terms of the abstract
  information content that needs to be specified for defining \OM symbols. The current
  implementation is thus just one possible encoding of this abstract model.
\item The \emph{CDUses} element is not part of this information model and has been made
  optional and deprecated in the reference encoding since it is trivial to extract its
  content automatically from the CD.
\item A CD may now, optionally, define its cdbase.
\item A CD symbol definition may now, optionally, define its role.
\item An FMP may, optionally, have a \lstinline|kind| attribute for use in classifying
  different kinds of definitions.  The details of how this attribute is used are not
  mandated by the standard.
\item The XML encoded Content Dictionaries now use elements from the namespace
  \url{http://www.openmath.org/OpenMathCD}.
\end{itemize}
\end{appendix}

%%% Local Variables:
%%% mode: latex
%%% TeX-master: "omstd20"
%%% End:
