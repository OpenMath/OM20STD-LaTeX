\chapter{\OM Encodings}\label{cha_enco}

In this chapter, two encodings are defined that map between \OM objects and byte streams.
These byte streams constitute a low level representation that can be easily exchanged
between processes (via almost any communication method) or stored and retrieved from
files.

The first encoding is a character-based encoding in \XML format.  In previous versions of
the \OM Standard this encoding was a restricted subset of the full legal \XML syntax.  In
this version, however, we have removed all these restrictions so that the earlier encoding
is a strict subset of the existing one.  The \XML encoding can be used, for example, to
send \OM objects via e-mail, cut-and-paste, etc. and to embed \OM objects in \XML
documents or to have \OM objects processed by \XML-aware applications.

The second encoding is a binary encoding that is meant to be used when the compactness of
the encoding is important (inter-process communications over a network is an example).

Note that these two encodings are sufficiently different for
auto-detection to be effective: an application reading the bytes can
very easily determine which encoding is used.

\section{The \XML Encoding}\label{sec_xml}

This encoding has been designed with two main goals in mind:
\begin{enumerate}
\item to provide an encoding that uses common character sets (so that it can easily be
  included in most documents and transport protocols) and that is both readable and
  writable by a human.
\item to provide an encoding that can be included (embedded) in \XML documents or
  processed by \XML-aware applications.
\end{enumerate}

\subsection{A Schema for the \XML Encoding}\label{ssec_xml}

The \XML encoding of an \OM object is defined by the Relax NG schema \cite{RELAX} given
below.  Relax NG has a number of advantages over the older XSD Schema format \cite{XSD},
in particular it allows for tighter control of attributes and has a modular, extensible
structure.  Although we have made the \XML form, which is given in \ref{app_openmath.rng},
normative, it is generated from the compact syntax given below.  It is also very easy to
restrict the schema to allow a limited set of \OM symbols as described in
\ref{app_relaxrestricted}.

Standard tools exist for generating a DTD or an XSD schema from a Relax NG Schema.
Examples of such documents are given in \ref{app_xsd}, respectively.

\lstinputlisting{openmath2.rnc}

\textbf{Note:} This schema specifies names as being of the \lstinline|xsd:NCName| type. At
the time of writing, W3C Schema types are defined in terms of XML 1 \cite{xml_98}.  This
limits the characters allowed in a name to a subset of the characters available in Unicode
2.0, which is far more restrictive than the definition for an \OM name given in
\ref{sec_names}.  It is expected that W3C Schema types will be augmented to match the new
XML 1.1 recommendation \cite{xml_04}, but for portability reasons applications should
avoid using the new XML 1.1 name characters unless they are absolutely required.  The XML
1.1 specification has a useful appendix giving advice on good strategies to use when
naming identifiers.

\subsection{Informal description of the \XML Encoding}\label{sec_xml-desc}

An encoded \OM object is placed inside an \lstinline|OMOBJ| element.  This 
element can contain the elements (and integers) described above.
 It can take an optional
\lstinline|version| (\XML) attribute which indicates to
which version of the \OM standard it conforms.  In previous versions of
this standard this attribute did not exist, so any \OM object without
such an attribute must conform to version 1 (or equivalently 1.1) of the
\OM standard.  Objects which conform to the description given in this
document should have \lstinline|version="2.0"|.

We briefly discuss the \XML encoding for each type of \OM object starting from the basic
objects.

\begin{description}
\item[Integers] are encoded using the
  \lstinline|OMI| element around the sequence of their
  digits in base 10 or 16 (most significant digit first).  White space
  may be inserted between the characters of the integer representation,
  this will be ignored.  After ignoring white space, integers written in
  base 10 match the regular expression
  \lstinline|-?[0-9]+|.  Integers written in base 16 match
  \lstinline|-?x[0-9A-F]+|.  The integer 10 can be thus
  encoded as \lstinline|<OMI> 10 </OMI>| or as
  \lstinline|<OMI> xA </OMI>| but neither
  \lstinline|<OMI> +10 </OMI>| nor
  \lstinline|<OMI> +xA </OMI>| can be used.

  The negative integer $-120$ can be encoded
  as either as decimal \lstinline|<OMI> -120</OMI>| or as hexadecimal 
  \lstinline|<OMI>-x78 </OMI>|.
\item[Symbols] are encoded using the \lstinline|OMS| element. This element has three
  (\XML) attributes \lstinline|cd|, \lstinline|name|, and \lstinline|cdbase|. The value
  of \lstinline|cd| is the name of the Content Dictionary in which the symbol is defined
  and the value of \lstinline|name| is the name of the symbol.  The optional
  \lstinline|cdbase| attribute is a URI that can be used to disambiguate between two
  content dictionaries with the same name.  If a symbol does not have an explicit
  \lstinline|cdbase| attribute, then it inherits its \lstinline|cdbase| from the first
  ancestor in the \XML tree with one, should such an element exist.  In this document we
  have tended to omit the \lstinline|cdbase| for clarity.
  
  For example:
\begin{lstlisting}
<OMS cdbase="http://www.openmath.org/cd" cd="transc1" name="sin"/>
\end{lstlisting}
  is the encoding of the symbol named \lstinline|sin| in the Content Dictionary named
  \lstinline|transc1|, which is part of the collection maintained by the \OM Society.

  As described in \ref{sec_names}, the three attributes of the \lstinline|OMS| can be used
  to build a URI reference for the symbol, for use in contexts where URI-based referencing
  mechanisms are used.  For example the URI for the above symbol is
  \url{http://www.openmath.org/cd/transc1\#sin}.

  Note that the role attribute described in \ref{sec_roles} is contained in the Content
  Dictionary and is not part of the encoding of a symbol, also the \lstinline|cdbase|
  attribute need not be explicit on each \lstinline|OMS| as it is inherited from any
  ancestor element.
\item[Variables] are encoded using the \lstinline|OMV| element, with only one (\XML)
  attribute, \lstinline|name|, whose value is the variable name. For instance, the
  encoding of the object representing the variable $x$ is: \lstinline|<OMV name="x"/>|
\item[Floating-point numbers] are encoded using the \lstinline|OMF| element that has
  either the (\XML) attribute \lstinline|dec| or the (\XML) attribute \lstinline|hex|. The
  two (\XML) attributes cannot be present simultaneously. The value of \lstinline|dec| is
  the floating-point number expressed in base 10, using the common syntax:

\begin{lstlisting}
(-?)([0-9]+)?("."[0-9]+)?([eE](-?)[0-9]+)?
\end{lstlisting}

or one of the special values: INF, -INF or NaN.

The value of \lstinline|hex| is a base 16 representation of the 64 bits of the
\acronym{ieee} Double.  Thus the number represents mantissa, exponent, and sign from
lowest to highest bits using a least significant byte ordering.  This consists of a string
of 16 digits \lstinline|0|-\lstinline|9|, \lstinline|A|-\lstinline|F|.
  

For example, both \lstinline|<OMF dec="1.0e-10"/>| and 
\lstinline|<OMF hex="3DDB7CDFD9D7BDBB"/>|
are valid representations of the floating point number $1\times 10^{-10}$.
 
The symbols \lstinline|INF|, \lstinline|-INF| and \lstinline|NaN| represent positive and
negative infinity, and \emph{not a number} as defined in \cite{ieee754_85}.  Note that
while infinities have a unique representation, it is possible for NaNs to contain extra
information about how they were generated and if this informations is to be preserved then
the hexadecimal representation must be used.  For example
\lstinline|<OMF hex="FFF8000000000000"/>| and \lstinline|<OMFhex="FFF8000000000001"/>| are
both hexadecimal representations of NaNs.
\item[Character strings] are encoded using the \lstinline|OMSTR| element.  Its
  content is a Unicode text. Note that as always in \XML the characters \lstinline|<| and
  \lstinline|\&| need to be represented by the entity references \lstinline|\&lt;|
  and \lstinline|\&amp;| respectively.
\item[Bytearrays] are encoded using the \lstinline|OMB| element. Its content is
  a sequence of characters that is a base64 encoding of the data.  The base64 encoding is
  defined in \acronym{rfc} 2045 \cite{rfc2045}.  Basically, it represents an arbitrary
  sequence of octets using 64 \textquote{digits} (\lstinline|A| through \lstinline|Z|,
  \lstinline|a| through \lstinline|z|, \lstinline|0| through \lstinline|9|,
  \lstinline|+| and /, in order of increasing value). Three octets are represented as
  four digits (the \lstinline|=| character is used for padding at the end of the
  data). All line breaks and carriage return, space, form feed and horizontal tabulation
  characters are ignored. The reader is referred to \cite{rfc2045} for more detailed
  information.
\end{description}
 
\begin{description}
\item[Applications] are encoded using the \lstinline|OMA| element. The
  application whose head is the \OM object $e_0$ and whose arguments
  are the \OM objects $e_1$, \ldots, $e_n$ is encoded as \lstinline|<OMA>|
  $C_0$ $C_1$\ldots $C_n$ \lstinline|</OMA>| where $C_i$ is the encoding of
  $e_i$.


For example, $\application{sin,x}$ is encoded as:
\begin{lstlisting}
<OMA>  
  <OMS cd="transc1" name="sin"/> 
  <OMV name="x"/>  
</OMA>
\end{lstlisting}
  provided that the symbol \lstinline|sin| is defined to be a function
  symbol in a Content Dictionary named \lstinline|transc1|.
\item[Binding] is encoded using the \lstinline|OMBIND| element.  The binding by the \OM
  object $b$ of the \OM variables $x_1$, $x_2$,\ldots, $x_n$ in the object $c$ is encoded
  as \lstinline|<OMBIND>| $B$ \lstinline|<OMBVAR>| $X_1$,\ldots, $X_n$
  \lstinline|</OMBVAR>| $C$ \lstinline|</OMBIND>| where $B$, $C$, and $X_i$ are the
  encodings of $b$, $c$ and $x_i$, respectively.

  For instance the encoding of $\binding{\lambda,x,\application{\sin,x}}$is:
\begin{lstlisting}
<OMBIND>
  <OMS cd="fns1" name="lambda"/>  
  <OMBVAR><OMV name="x"/></OMBVAR>  
  <OMA>
    <OMS cd="transc1" name="sin"/> 
    <OMV name="x"/>  
  </OMA>
</OMBIND>
\end{lstlisting}
  
Binders are defined in  Content Dictionaries, in particular,
  the symbol \lstinline|lambda| is defined in the Content Dictionary
  \lstinline|fns1| for functions over functions.
\item[Attributions] are encoded using the \lstinline|OMATTR| element.  If
  the \OM object $e$ is attributed with ($s_1$, $e_1$), \ldots, 
  ($s_n$, $e_n$) pairs (where $s_i$ are the attributes), it is encoded
  as \lstinline|<OMATTR>| \lstinline|<OMATP>| $S_1$ $C_1$ \ldots $S_n$ $C_n$ \lstinline|</OMATP>| $E$ \lstinline|</OMATTR>| where $S_i$ is the encoding of the
  symbol $s_i$, $C_i$ of the object $e_i$ and $E$ is the encoding of
  $e$.


Examples are the use of attribution to decorate a group by its
  automorphism group:
\begin{lstlisting}
<OMATTR>    
  <OMATP>
    <OMS cd="groups" name="automorphism_group" />  
    [..group-encoding..] 
  </OMATP>  
  [..group-encoding..] 
</OMATTR>
\end{lstlisting}
or to express the type of a variable:
\begin{lstlisting}
<OMATTR>    
  <OMATP>
    <OMS cd="ecc" name="type" /> 
    <OMS cd="ecc" name="real" />
  </OMATP> 
  <OMV name="x" />
</OMATTR>
\end{lstlisting}

A special use of attributions is to associate non-\OM data with an \OM object.  This is
done using the \lstinline|OMFOREIGN| element.  The children of this element must be
well-formed \XML.  For example the attribution of the \OM object $\sin(x)$ with its
representation in Presentation MathML is:
\begin{lstlisting}
<OMATTR>
  <OMATP>
    <OMS cd="annotations1" name="presentation-form"/>  
    <OMFOREIGN encoding="MathML-Presentation">
      <math xmlns="http://www.w3.org/1998/Math/MathML">
        <mi>sin</mi><mfenced><mi>x</mi></mfenced>
      </math>
    </OMFOREIGN>  
  </OMATP>
  <OMA>
   <OMS cd="transc1" name="sin"/> 
   <OMV name="x"/>  
  </OMA>
</OMATTR>
\end{lstlisting}
Of course not everything has a natural XML encoding in this way and
often the contents of a \lstinline|OMFOREIGN| will just
be data or some kind of encoded string.  For example the attribution
of the previous object with its LaTeX representation could be achieved
as follows:
\begin{lstlisting}
<OMATTR>
  <OMATP>
    <OMS cd="annotations1" name="presentation-form"/>  
    <OMFOREIGN encoding="text/x-latex">\sin(x)</OMFOREIGN>  
  </OMATP>
  <OMA>
    <OMS cd="transc1" name="sin"/> 
    <OMV name="x"/>  
  </OMA>
</OMATTR>
\end{lstlisting}
For a discussion on the use of the \lstinline|encoding|
attribute see \ref{sec_compl_omforeign}.
\item[Errors] are encoded using the \lstinline|OME| element. The error whose symbol is $s$
  and whose arguments are the \OM objects or \OM derived objects $e_1$, \ldots, $e_n$ is
  encoded as \lstinline|<OME>| $C_s$ $C_1$\ldots $C_n$ \lstinline|</OME>| where $C_s$ is
  the encoding of $s$ and $C_i$ the encoding of $e_i$.

  If an \lstinline|aritherror| Content Dictionary contained a \lstinline|DivisionByZero|
  symbol, then the object
  $\error{DivisionByZero{\application{divide,x,0}}}$ would be encoded as follows:

\begin{lstlisting}
<OME>
  <OMS cd="aritherror" name="DivisionByZero"/>  
  <OMA>
    <OMS cd="arith1" name="divide" />
    <OMV name="x"/>  
    <OMI> 0 </OMI>
  </OMA> 
 </OME>
\end{lstlisting}

  If a \lstinline|mathml| Content Dictionary contained an \lstinline|unhandled_csymbol|
  symbol, then an \OM to MathML translator might return an error such as:
\begin{lstlisting}
<OME>
  <OMS cd="mathml" name="unhandled_csymbol"/>  
  <OMFOREIGN encoding="MathML-Content">
    <mathml:csymbol xmlns:mathml="http://www.w3.org/1998/Math/MathML/"
                    definitionURL="http://www.nag.co.uk/Airy#A">
      <mathml:mo>Ai</mathml:mo>
    </mathml:csymbol>
  </OMFOREIGN> 
 </OME>
\end{lstlisting}

  Note that it is possible to embed fragments of valid \OM inside an \lstinline|OMFOREIGN|
  element but that it cannot contain invalid \OM.  In addition, the arguments to an
  \lstinline|OMERROR| must be well-formed \XML.  If an application wishes to signal that
  the \OM it has received is invalid or is not well-formed then the offending data must be
  encoded as a string.  For example:
\begin{lstlisting}
<OME>
  <OMS cd="parser" name="invalid_XML"/>  
  <OMSTR>
    &ltOMA> <OMS name="cos" cd="transc1">
      <OMV name="v"> </OMA>
  </OMSTR> 
 </OME>
\end{lstlisting}
  Note that the \textquote{<} and \textquote{>} characters have been escaped as is usual in
  an \XML document.
\item[References] \OM integers, floating point numbers, character strings, bytearrays,
  applications, binding, attributions can also be encoded as an empty \lstinline|OMR|
  element with an \lstinline|href| attribute whose value is the value of a URI referencing
  an id attribute of an \OM object of that type.  The \OM element represented by this
  \lstinline|OMR| reference is a copy of the \OM element referenced \lstinline|href|
  attribute. Note that this copy is \emph{structurally equal}, but not identical to the
  element referenced. These URI refererences will often be relative, in which case they
  are resolved using the base URI of the document containing the \OM.

 For instance, the \OM object
\begin{lstlisting}
 <math id="nestedap" display="block">
   <mrow>
     <mi mathvariant="bold">application</mi>
     <mrow>
       <mo fence="true">(</mo>
       <mrow>
         <mi>f</mi>
         <mo separator="true">,</mo>
         <mi mathvariant="bold">application</mi>
         <mrow>
           <mo fence="true">(</mo>
           <mrow>
             <mi>f</mi>
             <mo separator="true">,</mo>
             <mi mathvariant="bold">application</mi>
             <mrow>
               <mo fence="true">(</mo>
               <mrow><mi>f</mi><mo separator="true">,</mo><mi>a</mi><mo separator="true">,</mo><mi>a</mi></mrow>
               <mo fence="true">)</mo>
             </mrow>
             <mo separator="true">,</mo>
             <mi mathvariant="bold">application</mi>
             <mrow>
               <mo fence="true">(</mo>
               <mrow><mi>f</mi><mo separator="true">,</mo><mi>a</mi><mo separator="true">,</mo><mi>a</mi></mrow>
               <mo fence="true">)</mo>
             </mrow>
             <mo fence="true">)</mo>
           </mrow>
           <mo separator="true">,</mo>
           <mi mathvariant="bold">application</mi>
           <mrow>
             <mo fence="true">(</mo>
             <mrow>
               <mi>f</mi>
               <mo separator="true">,</mo>
               <mi mathvariant="bold">application</mi>
               <mrow>
                 <mo fence="true">(</mo>
                 <mrow><mi>f</mi><mo separator="true">,</mo><mi>a</mi><mo separator="true">,</mo><mi>a</mi></mrow>
                 <mo fence="true">)</mo>
               </mrow>
               <mo separator="true">,</mo>
               <mi mathvariant="bold">application</mi>
               <mrow>
                 <mo fence="true">(</mo>
                 <mrow><mi>f</mi><mo separator="true">,</mo><mi>a</mi><mo separator="true">,</mo><mi>a</mi></mrow>
                 <mo fence="true">)</mo>
               </mrow>
               <mo fence="true">)</mo>
             </mrow>
           </mrow>
           <mo fence="true">)</mo>
         </mrow>
       </mrow>
     </mrow>
   </mrow>
 </math>
\end{lstlisting}
can be encoded in the \XML encoding as either one of the \XML encodings given in
\ref{fig_shared_vs_unshared} (and some intermediate versions as well).
\end{description}

\begin{figure}\centering
\caption{Shared vs. unshared representations}\label{fig_shared_vs_unshared}
    
\begin{lstlisting}
<OMOBJ version="2.0">         <OMOBJ version="2.0">
  <OMA>                         <OMA>
    <OMV name="f"/>               <OMV name="f"/> 
    <OMA>                         <OMA id="t1">
      <OMV name="f"/>               <OMV name="f"/>
      <OMA>                         <OMA id="t11">
        <OMV name="f"/>               <OMV name="f"/>
        <OMV name="a"/>               <OMV name="a"/>
        <OMV name="a"/>               <OMV name="a"/>
      </OMA>                        </OMA>
      <OMA>                         <OMR href="#t11"/>
        <OMV name="f"/>
        <OMV name="a"/> 
        <OMV name="a"/>
      </OMA>                                
    </OMA>                      </OMA>
    <OMA>                       <OMR href="#t1"/>
      <OMV name="f"/>
      <OMA>
        <OMV name="f"/>
        <OMV name="a"/>
        <OMV name="a"/>
      </OMA>
      <OMA>
        <OMV name="f"/>
        <OMV name="a"/>
        <OMV name="a"/>
      </OMA>
    </OMA>
  </OMA>
</OMOBJ>                     </OMOBJ>
\end{lstlisting}
\end{figure}

\subsection{Some Notes on References}\label{sec_references}

We say that an \OM element dominates all its children and all elements
they dominate. An \lstinline|OMR| element dominates its target,
i.e. the element that carries the \lstinline|id| attribute pointed to
by the \lstinline|xref| attribute. For instance in the representation
in \ref{fig_shared_vs_unshared}, the
\lstinline|OMA| element with \lstinline|id="t1"| and
also the second \lstinline|OMR| dominate the
\lstinline|OMA| element with \lstinline|id="t11"|.

\subsubsection{An Acyclicity Constraint}\label{sec_acyclicity}

The occurrences of the \lstinline|OMR| element must obey the following global
\emph{acyclicity constraint}: An \OM element may not dominate itself.

Consider for instance the following (illegal) \XML representation
\begin{lstlisting}
<OMOBJ version="2.0">
  <OMA id="foo">
    <OMS cd="arith1" name="divide"/>
    <OMI>1</OMI>
    <OMA>
       <OMS cd="arith1" name="plus"/>
       <OMI>1</OMI>
       <OMR xref="foo"/>
    </OMA> 
  </OMA>
</OMOBJ>
\end{lstlisting}

Here, the \lstinline|OMA| element with
\lstinline|id="foo"| dominates its third child, which dominates the
\lstinline|OMR| element, which dominates its target: the element with
\lstinline|id="foo"|. So by transitivity, this element dominates itself, and
by the acyclicity constraint, it is not the \XML representation of an \OM
element. Even though it could be given the interpretation of the continued fraction
\begin{lstlisting}
<math display="block">
 <mfrac>
   <mn>1</mn>
   <mrow>
     <mn>1</mn>
     <mo>+</mo>
     <mfrac>
       <mn>1</mn>
       <mrow>
         <mn>1</mn>
         <mo>+</mo>
         <mfrac><mn>1</mn><mi>...</mi></mfrac>
       </mrow>
     </mfrac>
   </mrow>
 </mfrac>
</math>
\end{lstlisting}
this would correspond to an infinite tree of applications, which is not admitted by the
structure of \OM objects described in \ref{cha_obj}.

Note that the acyclicity constraints is not restricted to such simple cases, as the
example in \ref{fig_sharing_between} shows.

\begin{figure}\centering
\caption{Sharing between \OM objects (A cycle of order 2).}\label{fig_sharing_between}
\begin{lstlisting}
% <OMOBJ version="2.0">                   <OMOBJ version="2.0">
  <OMA id="bar">                         <OMA id="baz">
    <OMS cd="arith1" name="plus"/>         <OMS cd="arith1" name="plus"/>
    <OMI>1</OMI>                           <OMI>1</OMI>
    <OMR xref="baz"/>                      <OMR xref="bar"/>
  </OMA>                                 </OMA>
</OMOBJ>                               </OMOBJ>
\end{lstlisting}
\end{figure}

Here, the \lstinline|OMA| with \lstinline|id="bar"| dominates its third child, the
\lstinline|OMR| with \lstinline|xref="baz"|, which dominates its target \lstinline|OMA|
with \lstinline|id="baz"|, which in turn dominates its third child, the \lstinline|OMR|
with \lstinline|xref="bar"|, this finally dominates its target, the original
\lstinline|OMA| element with \lstinline|id="bar"|. So this pair of \OM objects violates
the acyclicity constraint and is not the \XML representation of an \OM object.

\subsubsection{Sharing and Bound Variables}\label{sec_sharing_bvars}

Note that the \lstinline|OMR| element is a \emph{syntactic} referencing mechanism: an
\lstinline|OMR| element stands for the exact \XML element it points to. In particular,
referencing does not interact with binding in a semantically intuitive way, since it
allows for variable capture. Consider for instance the following \XML representation:
\begin{lstlisting}
 <OMBIND id="outer">
  <OMS cd="fns1" name="lambda"/>
  <OMBVAR><OMV name="X"/></OMBVAR>
  <OMA>
    <OMV name="f"/>
    <OMBIND id="inner">
      <OMS cd="fns1" name="lambda"/>
      <OMBVAR><OMV name="X"/></OMBVAR>
      <OMR id="copy" href="#orig"/>
    </OMBIND>
    <OMA id="orig"><OMV name="g"/><OMV name="X"/></OMA>
  </OMA>
</OMBIND>
\end{lstlisting}
it represents the \OM object
\[\binding{\lambda,X,\application{f,\binding{\lambda,X,\application{g,X},\application{g,X}}}}\]
which has two sub-terms of the form $\application{g,X}$, one with \lstinline|id="orig"|
(the one explicitly represented) and one with \lstinline|id="copy"|, represented by the
\lstinline|OMR| element. In the original, the variable $X$ is bound by the \emph{outer}
\lstinline|OMBIND| element, and in the copy, the variable $X$is bound by the \emph{inner}
\lstinline|OMBIND| element. We say that the inner \lstinline|OMBIND| has captured the
variable $X$.

It is well-known that variable capture does not conserve semantics. For instance, we could
use $\alpha$-conversion to rename the inner occurrence of $X$ into, say, $Y$ arriving at
the (same) object
\[\binding{\lambda,X,\application{f,\binding{\lambda,Y,\application{g,Y},\application{g,X}}}}\]
Using references that capture variables in this way can easily lead to representation
errors, and is not recommended.

\subsection{Embedding \OM in \XML Documents}\label{xmldoc}
     
The above encoding of \XML encoded \OM specifies the grammar to be used in files that
encode a single \OM object, and specifies the character streams that a conforming \OM
application should be able to accept or produce.


When embedding \XML encoded \OM objects into a larger \XML document one may wish, or need,
to use other \XML features. For example use of extra \XML attributes to specify \XML
Namespaces~\cite{xmlns} or \lstinline|xml:lang| attributes to specify the language used
in strings~\cite{xml_04}.

If such \XML features are used then the \XML application controlling the document must, if
passing the \OM fragment to an \OM application, remove any such extra attributes and must
ensure that the fragment is encoded according to the schema specified above.

\section{The Binary Encoding}\label{sec_binary}

The binary encoding was essentially designed to be more compact than the \XML encodings,
so that it can be more efficient if large amounts of data are involved. For the current
encoding, we tried to keep the right balance between compactness, speed of encoding and
decoding and simplicity (to allow a simple specification and easy implementations).

\subsection{A Grammar for the Binary Encoding}\label{sec_binary_grammar}

\def\abyte{[\_]}\def\fourbytes{\ensuremath{\{\_\}}}
\begin{figure}\centering\footnotesize
\begin{center}
\begin{tabular}{lcp{6cm}lcp{5cm}}
  start  &  $\longrightarrow$& [24] object [25] 
           &  $|$ &  [24+64] [$m$]  [$n$] object [25]\\
   object  & $\longrightarrow$& basic
      & $|$ &  compound &\\
      & $|$ & cdbase
      & $|$ & foreign \\
      & $|$ & reference &\\
    basic & $\longrightarrow$ &  integer  
      & $|$ &  float \\
      & $|$ &  variable 
      & $|$ &  symbol   \\
      & $|$ &  string 
      & $|$ &  bytearray \\
    integer  & $\longrightarrow$&[1] \abyte 
      & $|$ & [1+64] [$n$] id:$n$ \abyte\\
      & $|$ &  [1+32] \abyte  & &  \\
      & $|$ &  [1+128] \fourbytes 
      &  $|$ & [1+64+128] {$n$} id:$n$ \fourbytes\\
      & $|$ &  [1+32+128] \fourbytes  &
      &  &\\
      & $|$ &  [2] [$n$] \abyte digits:$n$
      & $|$ & [2+64] [$n$] [$m$] \abyte digits:$n$ id:$m$\\
      & $|$ & [2+32] [$n$] \abyte digits:$n$ & 
      &  \\
      & $|$ & [2+128] {$n$} \abyte digits:$n$
      & $|$ & [2+64+128] {$n$} {$n$} \abyte digits:$n$ id:$n$\\
      & $|$ & [2+32+128] {$n$} \abyte digits:$n$
      & & \\
   float  & $\longrightarrow$& [3] \fourbytes\fourbytes 
       & $|$ & [3+64] [$n$] id:$n$ \fourbytes\fourbytes\\
       & &
       & $|$ & [3+64+128] {$n$} id:$n$ \fourbytes\fourbytes\\
    variable  & $\longrightarrow$& [5] [$n$] varname:$n$
       & $|$ & [5+64] [$n$] [$m$] varname:$n$ id:$m$\\
       & $|$ & [5+128] {$n$} varname:$n$
       & $|$ & [5+64+128] {$n$} {$m$} varname:$n$ id:$m$\\
    symbol & $\longrightarrow$& [8] [$n$] [$m$] cdname:$n$ symbname:$m$
       & $|$ & [8+64] [$n$] [$m$] [$k$] cdname:$n$ symbname:$m$ id:$k$\\
       & $|$ & [8+128] {$n$} {$m$} cdname:$n$ symbname:$m$
       & $|$ & [8+64+128] {$n$} {$m$} {$k$} cdname:$n$ symbname:$m$ id:$k$\\
    string  & $\longrightarrow$& [6] [$n$] bytes:$n$
       & $|$ & [6+64] [$n$] bytes:$n$\\
       & $|$ & [6+32] [$n$] bytes:$n$ & & \\
       & $|$ & [6+128] {$n$} bytes:$n$
       & $|$ & [6+64+128] {$n$} {$m$} bytes:$n$ id:$m$\\
       & $|$ & [6+32+128] {$n$} bytes:$n$
       & & \\
       & $|$ & [7] [$n$] bytes:$2n$ 
       & $|$ & [7+64] [$n$] [$m$]bytes:$n$id:$m$ \\
       & $|$ & [7+32] [$n$] bytes:$2n$
       & & \\
       & $|$ & [7+128] {$n$} bytes:$2n$
       & $|$ & [7+64+128] {$n$} {$m$} bytes:$2n$ id:$m$\\
       & $|$ & [7+32+128] {$n$} bytes:$2n$ &&\\
   bytearray  & $\longrightarrow$& [4] [$n$] bytes:$n$
       & $|$ & [4+64] [$n$] [$m$] bytes:$n$ id:$m$\\
       & $|$ & [4+32] [$n$] bytes:$n$ && \\
       & $|$ & [4+128] {$n$} bytes:$n$
       & $|$ & [4+64+128] {$n$} {$m$} bytes:$n$ id:$m$\\
       & $|$ & [4+32+128] {$n$} bytes:$n$ && \\
    cdbase & $\longrightarrow$& [9] [$n$] uri:$n$ 
       &&\\
       & $|$ & [9+128] {$n$} uri:$n$ && \\
    foreign &$\longrightarrow$& [12] [$n$] [$m$] bytes:$n$ bytes:$m$
       & $|$ & [12+64] [$n$] [$m$] [$k$] bytes:$n$ bytes:$m$ id:$k$ \\
       & $|$ & [12+32] [$n$] [$m$] bytes:$n$ bytes:$m$ 
       &    & \\
       & $|$ & [12+128] {$n$} {$m$} bytes:$n$ bytes:$m$
       & $|$ & [12+64+128] {$n$} {$m$} {$k$} bytes:$n$ bytes:$m$ id:$k$\\
       & $|$ & [12+32+128] {$n$} {$m$} bytes:$n$ bytes:$m$
       &    &\\
   compound & $\longrightarrow$& application
      & $|$ & binding \\
      & $|$ & attribution
      & $|$ & error \\
   application & $\longrightarrow$ & [16] object objects [17] 
       & $|$ & [16+64] [$m$] id:$m$ object objects [17]\\
       &&
       & $|$ & [16+64+128] {$m$} id:$m$ object objects [17]\\
    binding & $\longrightarrow$&[26] object bvars object [27] 
       & $|$ & [26+64] [$m$] id:$m$ object bvars object [27] \\
       & & 
       & $|$ & [26+64+128] {$m$} id:$m$ object bvars object [27]\\
    attribution & $\longrightarrow$&[18] attrpairs object [19] 
       & $|$ & [18+64] [$m$] id:$m$ attrpairs object [19]\\
       & & 
       & $|$ & [18+64+128] {$m$} id:$m$ attrpairs object [19]\\
     error &  $\longrightarrow$&[22] symbol objects [23] 
        & $|$ & [22+64] [$m$] id:$m$ symbol objects [23]\\
       & & 
      & $|$ & [22+64+128] {$m$} id:$m$ symbol objects [23]\\
     attrpairs  & $\longrightarrow$&[20] pairs [21] 
        & $|$ & [20+64] [$m$] id:$m$ pairs [21]\\
       & & 
      & $|$ & [20+64+128] {$m$} id:$m$ pairs [21]\\
    pairs  & $\longrightarrow$&symbol object &\\
      & $|$ & symbol object pairs &\\
    bvars  & $\longrightarrow$&[28] vars [29] 
      & $|$ & [28+64] [$m$] id:$m$ vars [29]\\
      & & 
      & $|$ & [28+64+128] {$m$} id:$m$ vars [29]\\
   vars  & $\longrightarrow$& \emph{empty}
      & $|$ & attrvar vars &\\
   attrvar  & $\longrightarrow$&variable &\\
      & $|$ & [18] attrpairs attrvar [19] 
      & $|$ & [18+64] [$m$] id:$m$ attrpairs attrvar [19]\\
      & & 
      & $|$ & [18+64+128] {$m$} id:$m$ attrpairs attrvar [19]\\
  objects  & $\longrightarrow$&\emph{empty} 
      & $|$ & object objects \\
  reference &$\longrightarrow$& internal\_reference
      & $|$ & external\_reference \\
  internal\_reference  & $\longrightarrow$&[30] \abyte
      & $|$ & [30+128] \fourbytes \\
  external\_reference  & $\longrightarrow$& [31] [$n$] uri:$n$
      & $|$ & [31+128] {$n$} uri:$n$
\end{tabular}
\end{center}
\caption{Grammar of the binary encoding of \OM objects.}\label{fig_bin-enc}
\end{figure}
  
\ref{fig_bin-enc} gives a grammar for the binary encoding (\textquote{start} is the start
symbol).

The following conventions are used in this section: [$n$] denotes a byte whose value is
the integer $n$ ($n$ can range from 0 to 255), {$m$} denotes four bytes representing the
(unsigned) integer $m$ in network byte order, \abyte denotes an arbitrary byte, \fourbytes denotes
an arbitrary sequence of four bytes.  Finally, \emph{empty} stands for the empty list of
tokens.

\emph{xxxx}:$n$, where \emph{xxxx} is one of \emph{symbname}, \emph{cdname},
\emph{varname}, \emph{uri}, \emph{id}, \emph{digits}, or \emph{bytes} denotes a sequence
of $n$ bytes that conforms to the constraints on \emph{xxxx} strings. For instance, for
\emph{symbname}, \emph{varname}, or \emph{cdname} this is the regular expression described
in \ref{sec_names}, for \emph{uri} it is the grammar for URIs in \cite{IETF2396}.

\subsection{Description of the Grammar}\label{sec_bin-desc}
  
An \OM object is encoded as a sequence of bytes starting with the begin object tag (values
24 and 88) and ending with the end object tag (value~25). These are similar to the
\lstinline|<OMOBJ>| and \lstinline|</OMOBJ>| tags of the \XML encoding. Objects with
start token [88] have two additional bytes $m$ and $n$ that characterize the version
($m.n$) of the encoding directly after the start token. This is similar to
\lstinline|<OMOBJ version="m.n">|.

The encoding of each kind of \OM object begins with a tag that is a single byte, holding a
\emph{token identifier} that describes the kind of object, two flags, and a status
bit. The identifier is stored in the first five bits (1 to 5). Bit 6 is used as a
\emph{status bit} which is currently only used for managing streaming of some basic
objects. Bits 7 and 8 are the \emph{sharing flag} and the \emph{long flag}. The sharing
flag indicates that the encoded object may be shared in another (part of an) object
somewhere else (see \ref{sec_sharing_references}). Note that if the sharing flag is set
(in the right column of the grammar in \ref{fig_bin-enc}, then the encoding includes a
representation of an identifier that serves as the target of a reference (internal with
token identifier 30 or external with token identifier 31). If the long flag is set, this
signifies that the names, strings, and data fields in the encoded \OM object are longer
than 255 bytes or characters.

The concept of structure sharing in \OM encodings and in particular the sharing bit in the
binary encoding has been introduced in \OM~2 (see section \ref{sec_sharing_references} for
details). The binary encoding in \OM~2 leaves the tokens with sharing flag 0 unchanged to
ensure \OM~1 compatibility. To make use of functionality like the version attribute on the
\OM object introduced in \OM~2, the tokens with sharing flag 1 should be used.


To facilitate the streaming of \OM objects, some basic objects (integers, strings,
bytearrays, and foreign objects) have variant token identifiers with the fifth bit
set. The idea behind this is that these basic objects can be split into packets. If the
fifth bit is not set, this packet is the final packet of the basic object. If the bit is
set, then more packets of the basic object will follow directly after this one. Note that
all packets making up a basic object must have the same token identifier (up to the fifth
bit). In \ref{fig_bin-enc_stream} we have represented an integer that is split up into
three packets.

Here is a description of the binary encodings of every kind of \OM object:

\begin{description}
\item[Integers] are encoded depending on how large they
      are. There are four possible formats.  Integers between -128 and 127 are
      encoded as the small integer tags (token
	identifier 1) followed by a single byte that is the 
      value of the integer (interpreted as a signed character). For
      example 16 is encoded as \lstinline|0x01 0x10|.  Integers between
      $-2^{31}$  (-2147483648) and $2^{31-1}$  (2147483647) are encoded as
      the small integer tag with the long flag set followed by the integer
      encoded in little endian format in four bytes (network byte order:
      the most significant byte comes first). For example, 128 is encoded
      as \lstinline|0x81| \lstinline|0x00000080|.  The most
      general encoding begins 
      with the big integer tag (token identifier 2) with the long flag set
      if the number of bytes in the encoding of the digits is greater or
      equal than 256. It is followed by the length (in bytes) of the
      sequence of digits, encoded on one byte (0 to 255, if the long flag
      was not set) or four bytes (network byte order, if the long flag was
      set).  It is then followed by a byte describing the sign and the
      base.  This 'sign/base' byte is \lstinline|+|
      (\lstinline|0x2B|) or \lstinline|-|
      (\lstinline|0x2D|) for the sign or-ed with the base mask bits
      that can be \lstinline|0| for base 10 
      or \lstinline|0x40| for base 16 or
	\lstinline|0x80| for \textquote{base 256}.  It is
      followed by the 
      sequence of digits (as 
      characters for bases 10 and 16 as in the \XML
	encoing, and as bytes for base 256) in their natural
      order.  For example, the decimal
	number 8589934592
      ($2^{33}$) is encoded  as
\begin{lstlisting}
0x02 0x0A 0x2B 0x38 0x35 0x38 0x39 0x39 0x33 0x34 0x35 0x39 0x32
\end{lstlisting}
      and the
      hexadecimal number
      \lstinline|xffffff1| is 
      encoded as 
\begin{lstlisting}
  0x02 0x08 0x6b 0x66 0x66 0x66 0x66 0x66 0x66 0x66 0x31
\end{lstlisting}
in the base 16 character encoding and as 
\begin{lstlisting}
0x02 0x04 0xFF 0xFF 0xFF 0xFI
\end{lstlisting}
in the byte encoding (base 256).

Note that it is permitted to encode a \textquote{small} integer in any \textquote{bigger}
format.

To splice sequences of integer packets into integers, we have to consider three cases: In
the case of token identifiers 1, 33, and 65 the sequence of packets is treated as a
sequence of integer digits to the base of $2^7$ (most significant first). The case of
token identifiers 129, 161, and 193 is analogous with digits of base $2^{31}$. In the case
of token identifiers 2, 34, 66, 130, 162, and 194 the integer is assembled by
concatenating the string of decimal digits in the packets in sequence order (which
corresponds to most significant first).  Note that in all cases only the sequence-initial
packet may contain a signed integer. The sign of this packet determines the sign of the
overall integer.


\begin{figure}\centering
\begin{tabular}{llllll}
  Byte &Hex &Meaning &  Byte &Hex &Meaning \\
  1 &22 &begin streamed big integer tag & 7 &2B &sign + (disregarded)\\
  2 &FF &255 digits in packet &  8 &... & the 255 digits as characters \\
  3 &2B &sign + & 9 &2 &begin final big integer tag\\
  4 &... & the 255 digits as characters & 10 &42 &68 digits in packet \\
  5 &22 &begin streamed big integer tag & 11 &2B &sign + (disregarded) \\
  6 &FF &255 digits in packet & 12 &... & the 68 digits as characters 
\end{tabular}
\caption{Streaming a large Integer in the Binary Encoding.}\label{fig_bin-enc_stream}
\end{figure}

\item[Symbols] are encoded as the symbol tags (token identifier 8) with the long flag set
  if the maximum of the length in bytes in the \acronym{utf-8} encoding of the Content
  Dictionary name or the symbol name is greater than or equal to 256. The symbol tag is
  followed by the length in bytes in the \acronym{utf-8} encoding of the Content
  Dictionary name, the symbol name, and the \lstinline|id| (if the shared bit was set) as
  a byte (if the long flag was not set) or a four byte integer (in network byte
  order). These are followed by the bytes of the \acronym{utf-8} encoding of the Content
  Dictionary name, the symbol name, and the \lstinline|id|.
\item[Variables] aare encoded using the variable tags (token identifiers 5) with the long
  flag set if the number of bytes in the \acronym{utf-8} encoding of the variable name is
  greater than or equal to 256.  Then, there is the number of characters as a byte (if the
  long flag was not set) or a four byte integer (in network byte order), followed by the
  characters of the name of the variable. For example, the variable x is encoded as
  \lstinline|0x05 0x01 0x78|.
  \item[Floating-point number] are encoded using the floating-point number tags (token
    identifier 3) followed by eight bytes that are the IEEE 754
    representation~\cite{ieee754_85}, most significant bytes first. For example, 0.1 is
    encoded as \lstinline|0x03 0x000000000000f03f|.
  \item[Character string] are encoded in two ways depending on whether, the string is
    encoded in \acronym{utf-16} or \acronym{iso-8859-1} (\acronym{latin-1}).  In the case
    of \acronym{latin-1} it is encoded as the one byte character string tags (token
    identifier 6) with the long flag set if the number of bytes (characters) in the string
    is greater than or equal to 256.  Then, there is the number of characters as a byte
    (if the length flag was not set) or a four byte integer (in network byte order),
    followed by the characters in the string. If the string is encoded in
    \acronym{utf-16}, it is encoded as the \acronym{utf-16} character string tags (token
    identifier 7) with the long flag set if the number of characters in the string is
    greater or equal to 256. Then, there is the number of \acronym{utf-16} units, which
    will be the number of characters unless characters in the higher planes of Unicode are
    used, as a byte (if the long flag was not set) or a four byte integer (in network byte
    order), followed by the characters (\acronym{utf-16} encoded Unicode).

    Sequences of string packets are assumed to have the same encoding for every
    packet. They are assembled into strings by concatenating the strings in the packets in
    sequence order.
  \item[Bytearrays] are encoded using the bytearray tags (token identifier 4) with the
    long flag set if the number elements is greater than or equal to 256. Then, there is
    the number of elements, as a byte (if the long flag was not set) or a four byte
    integer (in network byte order), followed by the elements of the arrays in their
    normal order.

    Sequences of bytearray packets are assembled into byte arrays by concatenating the
    bytearrays in the packets in sequence order.
  \item[Foreign Objects] are encoded using the foreign object tags (token identifier 12)
    with the long flag set if the number of bytes is greater than or equal to 256 and the
    streaming bit set for dividing it up into packets. Then, there is the number
    $n$ of bytes used to encode the encoding, and the number
    $m$ of bytes used to encode the foreign
    object. $n$ and $m$ are represented as a byte
    (if the long flag was not set) or a four byte integer (in network byte order). These
    numbers are followed by an $n$-byte representation of the encoding
    attribute and an $m$ byte sequence of bytes encoding the foreign
    object in their normal order (we call these the payload bytes). The encoding attribute
    is encoded in \acronym{utf-8}.

    Sequences of foreign object packets are assembled into foreign objects by
    concatenating the payload bytes in the packets in sequence order.

    Note that the foreign object is encoded as a stream of bytes, not a stream of
    characters. Character based formats (including XML based formats) should be encoded in
    \acronym{utf-8} to produce a stream of bytes to use as the payload of the foreign
    object.
  \item[cdbase scopes] are encoded using the token identifier 9. The purpose of these
    scoping devices is to associate a \lstinline|cdbase| with an object. The start token
    [9] (or [137] if the long flag is set) is followed by a single-byte (or 4-byte- if the
    long flag is set) number $n$ and then by a seqence of $n$ bytes that represent the
    value of the \lstinline|cdbase| attribute (a URI) in \acronym{utf-8} encoding. This
    is then followed by the binary encoding of a single object: the object over which this
    \lstinline|cdbase| attribute has scope.
  \item[Applications] are encoded using the application tags (token identifiers 16 and
    17). More precisely, the application of $E_0$ to $E_1$, \ldots, $E_n$ is encoded using
    the application tags (token identifier 16), the sequence of the encodings of $E_0$ to
    $E_n$ and the end application tags (token identifier 17).
  \item[Bindings] are encoded using the binding tags (token identifiers 26 and 27). More
    precisely, the binding by $B$ of variables $V_1$, \ldots $V_n$ in $C$ is encoded as
    the binding tag (token identifier 26), followed by the encoding of $B$, followed by
    the binding variables tags (token identifier 28), followed by the encodings of the
    variables $V_1$, \ldots, $V_n$, followed by the end binding variables tags (token
    identifier 29), followed by the encoding of $C$, followed by the end binding tags
    (token identifier 27).
  \item[Attributions] are encoded using the attribution tags (token identifiers 18 and
    19). More precisely, attribution of the object $E$ with ($E_1 S_1$, \ldots, $E_n S_n$)
    pairs (where $S_i$ are the attributes) is encoded as the attributed object tag (token
    identifier 18), followed by the encoding of the attribute pairs as the attribute pairs
    tags (token identifier 20), followed by the encoding of each symbol and value,
    followed by the end attribute pairs tag (token identifier 21), followed by the
    encoding of $E$, followed by the end attributed object tag (token identifier 19).
  \item[Errors] are encoded using the error tags (token identifiers 22 and 23). More
    precisely, $S_0$ applied to $E_1$,\ldots, $E_n$ is encoded as the error tag (token
    identifier 22), the encoding of $S_0$, the sequence of the encodings of $E_0$to $E_n$
    and the end error tag (token identifier 23).
\item[Internal References] are encoded using the internal reference tags [30] and [30+128]
  (the sharing flag cannot be set on this tag, since chains of references are not allowed
  in the \OM binary encoding) with long flag set if the number of \OM sub-objects in the
  encoded \OM is greater than or equal to 256. Then, there is the ordinal number of the
  referenced \OM object as a byte (if the long flag was not set) or a four byte integer
  (in network byte order).
\item[External References] are encoded using the external reference tags [31] and [31+128]
  (the sharing flag cannot be set on this tag, since chains of references are not allowed
  in the \OM binary encoding) with the long flag set if the number of bytes in the
  reference URI is greater than or equal to 256. Then, there is the number of bytes in the
  URI used for the external reference as a byte (if the long flag was not set) or a four
  byte integer (in network byte order), followed by the URI.
\end{description} 

\subsection{Example of Binary Encoding}\label{sec_bin_example}

As a simple
example of the binary encoding, we can consider the \OM object
\[\application{times,\application{plus,x,y},\application{plus,x,z}}\]

It is binary encoded as the sequence of bytes given in \ref{fig_bin-enc_ex}.

\begin{figure}\centering\footnotesize
\begin{tabular}{lllllllll}
	    Byte &Hex &Meaning &
	    Byte &Hex &Meaning &
	    Byte &Hex &Meaning\\
	    1 &58 &begin object tag &
	    19 &10 &begin application tag &
	    40 &10 &begin application tag\\
	    2 &2 & version 2.0 (major) &
	    20 &08 &symbol tag &
	    41 &48 &symbol tag (with share bit on) \\
	    3 &0 & version 2.0 (minor) &
	    21 &06 &cd length &
	    42 &01 &reference to second symbol seen (arith1:plus)\\
	    4 &10 &begin application tag &
	    22 &04 &name length &
	    43 &45 &variable tag (with share bit on)  \\
	    5 &08 &symbol tag &
	    23 &61 &a (cd name begin &
	    44 &00 &reference to first variable seen (x) \\
	    6 &06 &cd length  &
	    24 &72 &r  . &
	    45 &05 &variable tag \\
	    7 &05 &name length &
	    25 &69 &i  . &
	    46 &01 &name length\\
	    8 &61 &a (cd name begin &
	    26 &74 &t  . &
	    47 &7a &z (variable name) \\
	    9 &72 &r  . &
	    27 &68 &h  . &
	    48 &11 &end application tag\\
	    10 &69 &i  . &
	    28 &31 &1  .) &
	    49 &11 &end application tag \\
	    11 &74 &t  . &
	    29 &70 &p (symbol name begin &
	    50 &11 &end application tag\\
	    12 &68 &h  . &
	    30 &6c &l  . &
            &&\\               
	    13 &31 &1  .) &
	    31 &75 &u  .  &
             &&\\
	    14 &74 &t (symbol name begin &
	    32 &73 &s  .)  &
            &&\\
	    15 &69 &i  . &
	    33 &05 &variable tag &
            && \\
	    16 &6d &m  . &
	    34 &01 &name length  &
            && \\
	    17 &65 &e  . &
	    35 &78 &x (name)  &
            &&\\
	    18 &73 &s  .) &
	    36 &05 &variable tag  &
            && \\
	    &&&
	    37 &01 &name length  &
	    &&\\
            &&&
	    38 &79 &y (variable name)  &
	    &&\\
            &&&
	    39 &11 &end application tag  &
	    &&\\
\end{tabular}
\caption{A Simple example of the \OM binary encoding.}\label{fig_bin-enc_ex}
\end{figure}

\subsection{Sharing}\label{sec_both_sharing}
  
  
\OM~2 introduced a new sharing mechanism, described below.  First however we describe the
original \OM~1 mechanism.
  
\subsubsection{Sharing in Objects beginning with the identifier [24]}\label{sec_sharing}
    
This form of sharing is deprecated but included for backwards compatibility with \OM~1.
It supports the sharing of symbols, variables and strings (up to a certain length for
strings) within one object. That is, sharing between objects is not supported.  A
reference to a shared symbol, variable or string is encoded as the corresponding tag with
the long flag not set and the shared flag set, followed by a positive integer $n$ encoded
as one byte (0 to 255). This integer references the $n+1$-th such sharable sub-object
(symbol, variable or string up to 255 characters) in the current \OM object (counted in
the order they are generated by the encoding).  For example, \lstinline|0x48 0x01|
references a symbol that is identical to the second symbol that was found in the current
object.  Strings with 8 bit characters and strings with 16 bit characters are two
different kinds of objects for this sharing. Only strings containing less than 256
characters can be shared (i.e. only strings up to 255 characters).

  
  
  \subsubsection{Sharing with References (beginning with [24+64])}\label{sec_sharing_references}
    
  In the binary encoding specified in the last section (which we keep for compatibility
  reasons, but deprecate in favor of the more efficient binary encoding specified in this
  section) only symbols, variables, and short strings could be shared. In this section, we
  will present a second binary encoding, which shares most of the identifiers with the one
  in the last one, but handles sharing differently. This encoding is signaled by the
  shared object tag [88].

    
  The main difference is the interpretation of the sharing flag (bit 7), which can be set
  on all objects that allow it. Instead of encoding a reference to a previous occurrence
  of an object of the same type, it indicates whether an object will be referenced later
  in the encoding. This corresponds to the information, whether an \lstinline|id|
  attribute is set in the \XML encoding. On the object identifier (where sharing does not
  make sense), the shared flag signifies the encoding described here ([88]=[24+64]).
        
  Otherwise integers, floats, variables, symbols, strings, bytearrays, and constructs are
  treated exactly as in the binary encoding described in the last section.

    
  The binary encoding with references uses the additional reference tags [30] for (short)
  internal references, [30+128] for long internal references, [31] for (short) external
  references, [31+128] for long external references. Internal references are used to share
  sub-objects in the encoded object (see \ref{fig_bin-enc2} for an example) by referencing
  their position; external references allow to reference \OM objects in other documents by
  a URI.

  Identifiers [30+64] and [30+64+128] are not used, since they would encode references
  that are shared themselves. Chains of references are redundant, and decrease both space
  and time efficiency, therefore they are not allowed in the \OM binary encoding.

    
  References consist of the identifier [30] ([30+128] for long references) followed by a
  positive integer $n$ coded on one byte (4 bytes for long references). This integer
  references the $n+1$-th shared sub-object (one where the shared flag is set) in the
  current object (counted in the order they are generated in the encoding). For example
  \lstinline|Ox7E Ox01| references the second shared sub-object. \ref{fig_bin-enc2} shows
  the binary encoding of the object in \ref{fig_shared_vs_unshared} above.

    
\begin{figure}\centering\footnotesize
  \begin{tabular}{lllllllll}
    Byte &Hex &Meaning &
    Byte &Hex &Meaning &
    Byte &Hex &Meaning \\
    1 &58 &begin object tag &
    12 &50 &begin application tag (shared) &
    23 &1E &short reference \\
    2 &2 & version 2.0 (major) &
    13 &05 &variable tag &
    24 &00 &to the first shared object \\
    3 &0 & version 2.0 (minor) &
    14 &01 &variable length &
    25 &11 &end application tag \\
    4 &10 &begin application tag &
    15 &66 &f  (variable name) &
    26 &1E &short reference \\
    5 &05 &variable tag &
    16 &05 &variable tag &
    27 &00 &to the second shared object \\
    6 &01 &variable length &
    17 &01 &variable length &
    28 &11 &end application tag \\
    7 &66 &f  (variable name) &
    18 &61 &a  (variable name) \\
    8 &50 &begin application tag (shared) &
    19 &05 &variable tag \\
    9 &05 &variable tag &
    20 &01 &variable length \\
    10 &01 &variable length &
    21 &61 &a  (variable name)\\
    11 &66 &f  (variable name) & 
    22 &11 &end application tag
  \end{tabular}
  \caption{A binary encoding of the \OM object from \ref{fig_shared_vs_unshared}.}\label{fig_bin-enc2}
\end{figure}    

It is easy to see that in this binary encoding, the size of the encoding is $13+7(d-1)$
bytes, where $d$ is the depth of the tree, while a totally unshared encoding is
$8\ast2^d-8$ bytes (sharing variables saves up to 256 bytes for trees up to depth 8 and
wastes space for greater depths). The shared \XML encoding only uses $32d+29$ bytes, which
is more space efficient starting at depth 9.
    
Note that in the conversion from the \XML to the binary encoding the identifiers on the
objects are not preserved. Moreover, even though the \XML encoding allows references
across objects, as in \ref{fig_sharing_between}, the binary encoding does not (the binary
encoding has no notion of a multi-object collection, while in the \XML encoding this would
naturally correspond to e.g.~the embedding of multiple \OM objects into a single \XML
document).
    
Note that objects need not be fully shared (or shared at all) in the binary encoding with
sharing.

\subsection{Implementation Note}\label{sec_impl_note}
  
A typical implementation of the binary encoding comes in two parts. The first part deals
with the unshared encodings, i.e. objects starting with the identifier \lstinline|[24]|.
  
This part uses four tables, each of 256 entries, for symbol, variables, 8 bit character
strings whose lengths are less than 256 characters and 16 bit character strings whose
lengths are less than 256 characters.  When an object is read, all the tables are first
flushed. Each time a sharable sub-object is read, it is entered in the corresponding table
if it is not full. When a reference to the shared $i$-th object of a given type is read,
it stands for the $i$-th entry in the corresponding table. It is an encoding error if the
$i$-th position in the table has not already been assigned (i.e. forward references are
not allowed).  Sharing is not mandatory, there may be duplicate entries in the tables (if
the application that wrote the object chose not to share optimally).
  
The part for the shared representations of \OM objects uses an unbounded array for storing
shared sub-objects. Whenever an object has the shared flag set, then it is read and a
pointer to the generated data structure is stored at the next position of the
array. Whenever a reference of the form \lstinline|[30] [_]| is encountered, the array is
queried for the value at \lstinline|[_]| and analogously for \lstinline|[30+128] {_}|. 
Note that the application can decide to copy the value or share it among sub-terms
as long as it respects the identity conditions given by the tree-nature of the \OM
objects.  The implementation must take care to ensure that no variables are captured
during this process (see section \ref{sec_sharing_bvars}), and possibly have methods for
recovering from cyclic dependency relations (this can be done by standard loop-checking
methods).
  
Writing an object is simple. The tables are first flushed. Each time a sharable sub-object
is encountered (in the natural order of output given by the encoding), it is either
entered in the corresponding table (if it is not full) and output in the normal way or
replaced by the right reference if it is already present in the table.

\subsubsection{Relation to the \OM~1 binary encoding}\label{sec_relation_OM1_binary}

The \OM~2 binary encoding significantly extends the \OM~1 binary encoding to accommodate
the new features and in particular sharing of sub-objects. The tags and structure of the
\OM~1 binary encoding are still present in the current \OM binary encoding, so that binary
encoded \OM~1 objects are still valid in the \OM~2 binary encoding and correspond to the
same abstract \OM objects. In some cases, the binary encoding tags without the shared flag
can still be used as more compact representations of the objects (which are not shared,
and do not have an identifier).
  

As the binary encoding is geared towards compactness, \OM objects should be constructed so
as to maximise internal sharing (if computationally feasible). Note that since sharing is
done only at the encoding level, this does not alter the meaning of an \OM object, only
allows it to be represented more compactly.
  
\section{Summary}\label{sec_enc_summary}
  
The key points of this chapter are:
\begin{itemize}
\item The \XML encoding for \OM objects uses most common character sets.
\item The \XML encoding is readable, writable and can be embedded in most documents and
  transport protocols.
\item The binary encoding for \OM objects should be used when efficiency is a key
  issue. It is compact yet simple enough to allow fast encoding and decoding of objects.
\end{itemize}
  


%%% Local Variables:
%%% mode: latex
%%% TeX-master: "omstd20"
%%% End:
